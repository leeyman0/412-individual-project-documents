% Created 2021-05-03 Mon 22:32
% Intended LaTeX compiler: pdflatex
\documentclass[11pt]{article}
\usepackage[utf8]{inputenc}
\usepackage[T1]{fontenc}
\usepackage{graphicx}
\usepackage{grffile}
\usepackage{longtable}
\usepackage{wrapfig}
\usepackage{rotating}
\usepackage[normalem]{ulem}
\usepackage{amsmath}
\usepackage{textcomp}
\usepackage{amssymb}
\usepackage{capt-of}
\usepackage{hyperref}
\author{leeyman}
\date{\today}
\title{}
\hypersetup{
 pdfauthor={leeyman},
 pdftitle={},
 pdfkeywords={},
 pdfsubject={},
 pdfcreator={Emacs 27.2 (Org mode 9.4.4)}, 
 pdflang={English}}
\begin{document}

\tableofcontents

\section{Automata Football}
\label{sec:org928f8e5}
\begin{enumerate}
\item Good evening!
\item Hello, I am Stanley and this is my project, \emph{automata football}.
\item What is \emph{Automata Football}?
\begin{enumerate}
\item A simple quick online game based around Conway's Game of Life
\begin{enumerate}
\item Conway is an important mathematician, he devised this
\item It isn't really a game
\item Conway's Game of Life is a cellular automaton
\begin{enumerate}
\item What is a cellular automaton
\begin{enumerate}
\item Similar to a turing machine or lambda calculus, as it can compute things
\item The computation is turning the cells on or off
\end{enumerate}
\item An infinite grid of squares
\begin{enumerate}
\item squares (or cells) can either be on or off
\end{enumerate}
\item Square rules:
\begin{enumerate}
\item If a cell has more than three neighbors lit up, it is off next turn
\item If a cell has less than 2 neighbors lit up, it is off next turn
\item If a cell has exactly 3 neighbors lit up, it is on next turn
\item If a cell has exactly 2 neighbors lit up and it is on, then it stays on next turn
\end{enumerate}
\item People can make complex structures with this automaton
\end{enumerate}
\end{enumerate}
\item The goal is simple:
\begin{enumerate}
\item Engineer automata by clicking on the cells
\item Reach the opponent's end zone --- Touchdown
\item Secure your own end zone
\item Every square that gets lit up in the opponent's endzone is a point for you
\item Four turns
\item Most points wins
\end{enumerate}
\item Even though the goal is simple, the strategies have the capacity to be complex
\begin{enumerate}
\item Strategies stem from ambiguity
\begin{enumerate}
\item Can't see the opponent's side of the screen
\item Automata function
\end{enumerate}
\item Many different outcomes based on placement and and structure
\end{enumerate}
\end{enumerate}
\item Demonstration time
\begin{enumerate}
\item Technical Details while demonstration is going
\begin{enumerate}
\item Server written as a node.js webapp
\item Backend can run on Heroku
\item Client written in ES6 JS (2015)
\begin{enumerate}
\item Practically everything is based on JS, CSS, and HTML
\end{enumerate}
\item Utilizes WebSockets for communication between the server and client
\item Uses interactive SVGs for graphics
\end{enumerate}
\end{enumerate}
\item Incomplete features
\begin{enumerate}
\item Practice mode --- play automata football by yourself, Conway's game of life
\begin{enumerate}
\item I can probably do it, I just need more time
\end{enumerate}
\item Time limits to turns --- speed things up a bit
\begin{enumerate}
\item Requires a bit of communication through web sockets, and some computers are slower than others.
\end{enumerate}
\item Canvas client --- Just in case the SVG doesn't support some systems
\end{enumerate}
\end{enumerate}
\end{document}
